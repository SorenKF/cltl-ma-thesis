
\chapter{Latex tips} \label{ch:tips}
This chapter provides tips for using \LaTeX\ for writing your thesis, as well as more general tips for bibliographical references.

\section{Latex resources}
\href{https://www.latex-project.org}{Latex} is extremely well documented. 
The following resources will all give you an easy-to-step-in introduction, and an extensive reference to \LaTeX:
\begin{itemize}
\item the {\em Not so short introduction to \LaTeX 2e}: \url{https://tobi.oetiker.ch/lshort/lshort.pdf}
\item the \LaTeX\ wiki book: \url{https://en.wikibooks.org/wiki/LaTeX}
\item the Overleaf documentation: \url{https://www.overleaf.com/learn}
\end{itemize} 

Additionally, the \href{http://tug.ctan.org/info/symbols/comprehensive/symbols-a4.pdf}{\em Comprehensive \LaTeX\ symbols list} is worth a booktab.
Finally, \LaTeX\ has a lot of packages to offer for additional functionality, all stored on CTAN: \url{https://www.ctan.org}.


\section{Structure of a \LaTeX\ project}
Your thesis may run into more than 50 pages, and include as many pictures. It is therefore recommended to structure your \LaTeX\ thesis file into parts (a natural division consists in keeping a separate file for each chapter). \\

For this template, the file \texttt{mathesis.tex} is the main file. It links to content files stored in the \texttt{tex} folder. You will find there files for, e.g., the abstract and acknowledgments, but you can also add your chapter files.
Likewise, you can store images in the \texttt{img} folder.\\


\LaTeX\ documents can be included into one another using the \verb|\include| command: in the main file \texttt{mathesis.tex}, the assertion \verb|\frontchapter{Abstract}

% add your abstract here...

| looks for the file \texttt{tex/abstract.tex} and inserts its content into \texttt{mathesis.tex}.

\section{Citations}
\subsection{The natbib package}
The \texttt{natbib} package allows to refer to \texttt{BibTeX} bibliographical references and format them for insertion in a \LaTeX document. \texttt{BibTeX} bibliography items are stored in a \texttt{.bib} file.

For instance, the example bibliography \texttt{./bib/example.bib} contains two entries:
\begin{verbatim}
@inproceedings{sommerauer-etal-2019-towards,
        Address = {Wroclaw, Poland},
        Author = {Sommerauer, Pia and Fokkens, Antske and Vossen, Piek},
        Booktitle = {Proceedings of the 10th Global Wordnet Conference},
        Pages = {85--95},
        Title = {Towards Interpretable, Data-derived Distributional 
		Semantic Representations for Reasoning: A Dataset 
		of Properties and Concepts},
        Url = {https://clarin-pl.eu/dspace/handle/11321/718},
        Year = {2019},
        Bdsk-Url-1 = {https://clarin-pl.eu/dspace/handle/11321/718}}

@inproceedings{van-aggelen-etal-2019-larger,
        Address = {Turku, Finland},
        Author = {van Aggelen, Astrid and Fokkens, Antske and Hollink, 
		Laura and van Ossenbruggen, Jacco},
        Booktitle = {Proceedings of the 22nd Nordic Conference on 
		Computational Linguistics},
        Pages = {44--54},
        Publisher = {Link{\"o}ping University Electronic Press},
        Title = {A larger-scale evaluation resource of terms and 
		their shift direction for diachronic lexical semantics},
        Url = {https://www.aclweb.org/anthology/W19-6105.pdf},
        Year = {2019},
        Bdsk-Url-1 = {https://www.aclweb.org/anthology/W19-6105.pdf}}

\end{verbatim}

The first line of each entry provides a label for references: {\em sommerauer-etal-2019-towards}, {\em van-aggelen-etal-2019-larger}. These labels can be referred to in the \LaTeX\ document to provide formatted bibliographical references. \\

The two most commonly employed commands are \verb|\cite| (or equivalently \verb|\citet|) and \verb|\citep|. For instance, \verb|\citet{sommerauer-etal-2019-towards}| will appear as \citet{sommerauer-etal-2019-towards}, while \verb|\citep{sommerauer-etal-2019-towards}| will appear as \citep{sommerauer-etal-2019-towards}.

You can cite several papers with a single citation. For instance, the command
\verb|\citep{sommerauer-etal-2019-towards,van-aggelen-etal-2019-larger}| \hfill results 
in \citep{sommerauer-etal-2019-towards,van-aggelen-etal-2019-larger}.\\

See the \href{http://ctan.cs.uu.nl/macros/latex/contrib/natbib/natbib.pdf}{Natbib package documentation} or the usual \LaTeX\ references for more information. 

\subsection{Citing conventions}
It is convention to integrate the name of the authors in the text as much as possible, and to use \verb|\cite| as only the year of the reference is then parenthesized. The \verb|\citep| can be used when the name of the authors is not directly part of the sentence. 

For instance, you would use \verb|\cite| for ``the work of \cite{van-aggelen-etal-2019-larger}'', and \verb|\citep| for ``\ldots research on diachronic lexical semantics \citep{van-aggelen-etal-2019-larger}''. 

\subsection{Bibliography management}
We recommend that you use a bibliography management tool to edit \texttt{bib} files, like \href{https://bibdesk.sourceforge.io}{BibDesk} for Mac, or \href{https://www.jabref.org}{JabRef}.

This will provide you with a better overview of your bibliography as it grows, while facilitating the addition of new entries---this is as simple as copying the \texttt{bibtex} reference of an article and pasting it in the {\em bibtex source} field of your management tool.\\

Note that you can refer to distinct \texttt{bib} files in a \LaTeX\ document. Suppose for instance that you would like to keep apart references from the background chapter and from the other chapters, in \texttt{background.bib} and \texttt{research.bib}. You can collect both files with:
\verb|\bibliography{bib/background,bib/research}|.

Note however that \LaTeX\ will issue a warning if citations overlap between \texttt{bib} files.



 
